 \documentclass[a4paper]{article}
 
% PDF/A
\usepackage{hyperref}
\usepackage[a-1b]{pdfx}

%styling
\setlength\parindent{0pt}

% metadata
\title{Notions to Study Modern Computer Science}
\author{slbtty}
\date{November 10,2020}

\begin{document}
\maketitle
\section{basic symbols}

\subsection{A lambda to bind them all ($\lambda x.e$)}
$\lambda x.e$ meaning "the function that maps parameter x to expression e"\\

Today, we use $x \mapsto e$ as

\begin{itemize}
	\item $x \mapsto e\ or\ \lambda x.x^2 \Rightarrow$  the function $x^2$
	\item $(x,y)\mapsto x^2+y^2\ or\ \lambda x.\lambda y.x^2+y^2 \Rightarrow $ the function $x^2+y^2$
	\item $c\mapsto (x\mapsto cx^2)\ or\ \lambda c.\lambda x.cx^2\Rightarrow$ the family of functions $cx^2$
\end{itemize}


\subsubsection{Sets = predicates}
Membership:
\[
x\in A \equiv A\ x\ (application)
\]

As title said "sets = predicates", this one can be understood as "If A x is true, then x belongs to A". A x just a convection of lambda calculus' "apply".

Singleton:
\[
{x}\equiv \lambda y.x = y
\]
Binary union:
\[
A \cup B \equiv \lambda x. A\ x \vee B\ x
\]
General union (Arbitrary unions)
\[
\bigcup\!_{x\in A}B(x) \equiv \lambda y.\exists x, A\ x \wedge B\ x\ y
\]
B is a set for every x belongs to A. It is also called as \textit{infinitary union} since the B can be infinity large\\

If B is a set whose elements are sets, then x is an element of the union of B if and only if there is at least one element of A of B such that x is an element of A.

\section{Propositional calculus}
\subsection{examples}
\subsubsection{Modus Ponens}
Description: if $p$ then $q$; $p$; therefore $q$.
\[
((p \rightarrow q) \wedge p) \vdash q
\]

Alternatively, it can be written as

\[
\frac{p,p\Rightarrow q}{q}
\]

$\vdash$ and $\Rightarrow$ means "proves" or "implies".

\end{document}


